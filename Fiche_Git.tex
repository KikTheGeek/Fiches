\chapter{Git}

\section{Principle}

	Gits stores differences between the versions of the files at each commit. A project is a sequence of commit


	When you create a file commit it git stores the file. When you modify the file commit it git stores the difference 



\section{file statuses}

 
 	3 possible status for a file:

	\code{commited} added to a commit and stored by git

	\code{staged} currently staged for the next commit, but not yet stored by git

	\code{modified} modified since the last commit, but not staged yet


\section{steps}

	\subsection{initialize repository in a folder}
	
		\code{cd myFolder}

		\code{git init} generates a .git file 

	\subsection{tell git who you are: name + email}
	
		\code{git config --global user.email "your.email@domain.com"}
	
		\code{git config --global user.name "Your name"}

	\subsection{stage files for the next commit}

		\code{git add monfichier}


	\subsection{Commit}

		\code{git commit -m "Commit message here"} make a commit, -m is option for messages


\section{Link to Github}

	\code{git remote add origins https:\/\/github.com\/UserName\/FolderName.git} add a location for your repository on github

	\code{git push }


\section{Command recap}

	\code{git status} gives the status

	\code{git add monfichier} stages the file to the next commit

	\code{git diff} shows difference in each file with the last commit (q to exit)

	\code{git diff --staged} diff afer files are staged

	\code{git log} shows the commit history (q to exit)

	\code{git rm --cached monFichier} removes a file from git, cached keeps it on the drive