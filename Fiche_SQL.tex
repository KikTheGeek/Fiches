\chapter{SQL}

\section{Basics}


	\subsection{Data types}
		\code{FLOAT REAl INTEGER TEXT VARCHAR(225)...} some seem redundant, exist for compatibility

	\subsection{SELECT: get some columns}

\begin{lstlisting}
SELECT A,B
FROM MyDatabase;
\end{lstlisting}

	==> gets the colonne A and B from the DB

	==> ; indicates where the requests ends


	\subsection{WHERE: condition on the data}

\begin{lstlisting}
SELECT A,B
FROM MyDatabase
WHERE C < 0.5;
\end{lstlisting}

NB: see in AS for the use of HAVING when appropriate

	\subsection{Renaming}

		\subsubsection{On the fly}

\begin{lstlisting}
SELECT D1.A, D2.B as NewName FROM database1 D1, database2 D2 ...
\end{lstlisting}

		\subsubsection{Keyword As}

\begin{lstlisting}
SELECT A as NewName FROM ...
\end{lstlisting}

		output will have A name changed to NewName

		NB: WHERE does not work if you are creating an output with AS, instead, you'll need to use HAVING at the end

	\subsection{Missing values}

		\code{NULL} indicates a missing value

		supported by WHERE: \code{WHERE A is NULL}

		can be used in insert into, though some  columns have NOT NULL constraint

		\subsection{LIMIT: number max of results}

\begin{lstlisting}
SELECT A,B
FROM MyDatabase
WHERE C < 0.5
LIMIT 5;
\end{lstlisting}

	NB: very useful to have a look at a few rows

	\subsection{ORDER BY: orders data using given column}

\begin{lstlisting}
WHERE ...
ORDER BY D [ASC or DESC]
\end{lstlisting}

		can  order by several colum: \code{ORDER BY D [ASC or DESC], E [ASC or DESC] ...}

		NB: compulsory to write ASC  or DESC for each if one of the value is DESC

	\subsection{DISTINCT: eliminates duplicate}

		\code{SELECT DISTINCT A,B...} gets the unique pairs of values, can use with one or more columns
		
		NB: can be mixed with aggregations: \code{SELECT COUNT(DISTINCT A)}

	\subsection{Operators}

		\code{AND: WHERE A>0.5 AND B<=C}

		\code{OR: WHERE A>0.5 OR B<=C}

		\code{=}

		\code{!=}

	\subsection{Special characters}

		\code{*} allows you to get everything

	\subsection{Dates and time}

		SQL can handle date queries

		\code{SELECT ... WHERE time\_{}column > "2015-10-30 16:00"} ==> format is "yyyy-mm-dd", time as 24:00




\section{Maths}

	\subsection{aggregators}

		\code{SELECT COUNT(*) FROM} gets you the nber of rows

		\code{SELECT COUNT(Column) FROM} gets you the number non null value

		\code{SELECT MIN(Colums) or MAX} applies only to numerical columns

		\code{SUM}

		\code{AVG}

		NB: can do multiple stats at once SELECT SUM(..), AVG(..) etc

	\subsection{arithmetics}

		can do operations on numerical values and between columns

		\code{SELECT A /10.0 FROM...}

		\code{SELECT B*C FROM...}

		NB: integer and float types, operation with a float returns a float, if integer only returns an integer

	\subsection{ROUND}

		\code{SELECT ROUND(A,2)...} rounds A to 2 decimals

	\subsection{GROUP BY}

		allows to do group when there is a categorical column

\begin{lstlisting}
SELECT SUM(column)
FROM ...
GROUP BY ColumnWithCategories;
\end{lstlisting}

		NB: if no aggregation fuction will get the last row for each category


	\subsection{joining}

		\subsubsection{Implicit join (deprecated)}

\begin{lstlisting}
SELECT db1.col1, db1.col2, db2.col3
FROM db1, db2
ON db1.col3 = db2.col7
\end{lstlisting}

			The cross join just does \code{db1 $\times$ db2}, every row in db1 will be joined every row in db2, so if sizes are $n$ and $m$, output is $n \times m$. 

			\code{db1.col1} the doted notation allows to gain clarity and to avoid names conflicts.

		\subsubsection{explicit joins}

\begin{lstlisting}
SELECT db1.col1, db1.col2, db2.col3
FROM db1
INNER JOIN db2
ON db1.col3 = db2.col7
\end{lstlisting}


			\code{INNER JOIN} display rows from both table where the conditions match, rows that dont have a match in the other table are not displayed

			\code{OUTER JOIN} picks from both and puts Null when no match in other table

			\code{LEFT OUTER JOIN} shows all the left table and a matching rows from the right one if exixts, NULL otherwise

			\code{RIGHT OUTER JOIN} same but for right side


\section{Add and remove data}

	\subsection{Inserting new row}

		\code{INSERT INTO tablename VALUES (a, b, ....)}

	\subsection{updating existing row}

\begin{lstlisting}
UPDATE tableName
SET column1=value1, column2=value2, ...
WHERE column1=value3, column2=value4, ...
\end{lstlisting}

	\subsection{deleting some rows}

		\code{DELETE FROM tableName WHERE column1=value1, column2=value2, }...;

\subsection{add a column}
	
	\code{ALTER TABLE tableName ADD columnName dataType;}

	\subsection{remove a column}

\begin{lstlisting}
ALTER TABLE tableName
DROP COLUMN columnName;
\end{lstlisting}


	\subsection{create a table in an existing DB}

		\subsubsection{simple creation}

\begin{lstlisting}
CREATE TABLE dbName.tableName(
   column1 dataType1 PRIMARY KEY,
   column2 dataType2,
   column3 dataType3,
   ...
);
\end{lstlisting}

			NB: primary key is often id with type integer
			NB: can use IF NOT EXITS if fear of redudancy

		\subsubsection{Using a foreign key that points to another table}

\begin{lstlisting}
CREATE TABLE dbName.tableName(
   column1 dataType1 PRIMARY KEY,
   name text,
   country integer,
   worth float,
   FOREIGN KEY(country) REFERENCES otherTable(id)
);
\end{lstlisting}

		will points to the table \code{otherTable} via its primary key \code{id}


	\subsection{creating a database}
		
		\code{CREATE DATABASE dbName;}

	\subsection{deleting a database}
		\code{DROP DATABASE dbname;}




\section{Schema and database query}

	\subsection{PRAGMA: info about the DB}
		
		\code{PRAGMA table\_{}info(db)} returns info about the db (col names, types etc)

	\subsection{EXPLAIN: get the query plan}

		\code{EXPLAIN QUERY PLAN SELECT...} returns how the DB will run the query

\section{Indexing}

	\subsection{Single column}

		Where indexes exist, the programm uses them automatically to optimize queries.

		Indexes are automatically updated when rows or columns are changed.

		For a single column, the index will store ots value in order and the primary keys of the corresponding rows
		
		\code{CREATE INDEX index\_{}name ON table\_{}name(column\_{}name);}

		\code{CREATE INDEX IF NOT EXISTS index\_{}name ON table\_{}name(column\_{}name);}

\subsection{Multi column}

		\code{CREATE INDEX index\_{}name ON table\_{}name(column1\_{}name, column2\_{}name);}

		1st column will be the primary






\section{PostgreSQL command line}

	\subsection{basic stuff}
	
		\code{psql start}

		\code{psql -d dbname} starts and connects with the dbnames database

		\code{\textbackslash q} quit

		\code{\textbackslash ?} list of commands

		\code{\textbackslash l} list of databases

		\code{\textbackslash dt} tables in the current database

		\code{\textbackslash du} list of users

		can run any SQL query, can go next line with enter, only executed if ends with;

	\subsection{user management}

		\code{CREATE ROLE username} basic user wich cannot run queries

		\code{CREATE ROLE userName WITH LOGIN PASSWORD `password`;} user with access

		other options (after \code{WITH})

		\code{CREATEDB} can create DB

		\code{CREATEROLE} can create other users

		\code{SUPERUSER} make the user a superuser

	\subsection{ermission management}

		\code{GRANT SELECT, UPDATE, ..., ON tableName TO userName;}
		
		\code{GRANT ALL PRIVILEGES ON tablename TO username;} everything 

		\code{REVOKE... ON tableName FROM userName}; remove


