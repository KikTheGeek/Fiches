
\chapter{Python}

\section{Framework and useful}

Anaconda==> Python 3.5

	\code{iPython} script

	\code{Jupyter} machin en ligne


	\subsection{Jupyter notebook}
		\code{jupyter notebook} in cmd

		\code{\%matplotlib inline} makes the graphs in the page

	\subsection{Jupyter console}


		\subsubsection{commands}
			
			\code{ipython} open the jupyter console

			\code{exit} to quit

			\code{\%quickref} gives quick references

			\code{?} help (after a variable for information)

			\code{!command} allows to run shell commands

		\subsubsection{Magics}

			magics are keywords for the jupyter console

			\code{\%run} runs a python script

			\code{\%edit} opens a file editor, typed code is run when exiting

			\code{\%debug} opens a debug window

			\code{\%history}  last few commands

			\code{\%save} save history in a file

			\code{\%who} shows the variables

			\code{\%reset} reset variables

			\code{\%cpaste} opens an environment for pasting code (pasting doesnt work well in line)

			\url{http://ipython.readthedocs.io/en/stable/interactive/magics.html}
	

	\subsection{bash and co}

		\code{import os} work with the OS

		\code{virtualenv} cmd to create virtual env (solve versioning issues)

		\code{import sys} package to use argv

		\begin{lstlisting}
		__name__ == "__main__" 
		\end{lstlisting}  
		to avoid name mess when calling from cmd

	\subsection{Useful modules}

		\subsubsection{Fichiers}

			\code{f = open('monFichier.txt','r')}

			\subsubsection{csv}

				\code{fich  = csv.reader(f)}

				\code{fich = list(fich)}

		\subsubsection{others}
			
			\code{time}

			\code{datetime} (better for time zones)

			\code{sys} (system function such as argv)

			\code{json}

			\code{json.dumps} for python to json 

			\code{json.loads} for json to python



\section{Core data structures and logic}

	\subsection{Booleans and logic operators}

		\code{\|} or operator

		\code{\&} and operator



	\subsection{Regular expressions}

		\subsubsection{expressions}

			\code{"abc"}

			\code{"a.c"} a - any  letter  - c

			\code{"a[bde]c"} a - b or d or e - c

			\code{"a[d-h]c"} a -any letter between d and h -c

			\code{"\^{ }abc"} abc at the beginning

			\code{"abc\$"} abc at the end

			\code{[6-9]} any figure between 6 and 9

			\code{[6-9]\{3\}} repetition of 3 figures between 6 and 9


			\subsubsection{re module}
			
				\code{import re}

				\code{re.search(regex, string)} ==>returns \code{None} if nothing, match object otherwise

				\code{re.sub(pattern = , repl =, string = } we want to replace patter in string with repl



	\subsection{Dictionary}

		\code{dic = \{\}} creates an empty dictionnary

		\code{dic['mykey'] = value} add a value

		\code{value = dic['mykey']} access a value

		\code{val = dic.get('mykey', False)} returns the value for mykey, False if not in the dict. Default is \code{None}

	\subsection{List}

		\code{li = [] } creates an empty list

		\code{li = ["a", "b", "c", "d"]} 

		\code{li[2] = val} update the element

		\code{val = li[2]} access the element

		\code{li.append()}  add to a list at the end

		\code{l.delete(1)} delete elements with index 1, reindexes after it

		\code{len(li)} gives you the length

	\subsection{Tupples}
		
		\code{t = ()} creates an empty tupple

		\code{t = ('truc1', truc2')}

		\code{val = t[0]} access an element


	\subsection{enumerate}

		\code{for (i,item) in enumerate(MyItemList)} gets i as the index, item as the data

\section{Numpy module}

	\subsection{Lecture Fichiers}
		
\begin{lstlisting}
myData = numpy.genfromtxt("fichier.csv", delimiter=",", skipheader = True, dtype = "U75")}
\end{lstlisting}

	\subsection{ndarray}
		
		\code{import numpy as np}

		\code{numpy as nmyArray = np.array([1,2,3,4])}

		\code{np.add(array1, array2)} concatenation

		\code{np.sin(array1)} math functions for arrays

		\code{np.max(array2)}

		\code{myArray.shape} size

		can do some matlab:

		\code{myArray[:,1:3]}

		\code{myArray == 10} array of booleans

		\code{restriction: arrray = unique type}



\section{Pandas module}

	\subsection{load csv}

\begin{lstlisting}
df = pandas.read_csv("monFichier.csv") ==> DataFrame object
\end{lstlisting}

	\subsection{useful functions}
		
		\code{pandas.isnull()}

	\subsection{DataFrame object}

		\subsubsection{Columns and rows}
			
			\code{df["nomdelacolonne"]} gets a column as a series data type

			\code{df[ ["nomdelacolonne1", "nomdelacolonne2" ] ]} gets several columns as a DataFrame data type 

			\code{df["nouveau nom"] = mySerie} adds a column

			\code{.head(n)} n first rows

			\code{.columns} list of columns

			\code{.shape} sizes  ==> [0] for nber of row, [1] for col

			\code{.index} indexes

			\code{.loc[i]} gets a row as a series data type

			\code{.count()} with 0 or 1 nber of rows/columns

			NB: can do some matlab on loc too

			NB: loc works with index, may go off the rails when removing some rows, to access based on position, use iloc 

			\code{.dtype} series of the data type for each column

			\code{df.sort("nom de colonne", inplace=True, ascending=False)}

		\subsubsection{index}
			
			\code{.index} indexes

			\code{.reset\_{}index(drop = True)} reset the indexes to start at 0 and go one by one

			\code{.set\_{}index("nomcolonne", drop = True, Inplace = False)} returns new df indexed by the chosen column (drop = True means this column  is then removed in the new one)

		\subsubsection{drop}

			\code{.drop('nom de colonne', axis = 1)} kill a columns

			\code{.dropna()} removes rows with missing data

			\code{dropna.(subset=['nomcolonne1','nomcolonne2'])}

			\code{dropna.(axis=1)} removes colums with missing val


		\subsubsection{apply}
			
			\code{df2 = df.apply(mafonction)} function applied to each column. If function return a single value, then output is series, If function returns a series, output is df

			\code{ser = df.apply(mafonction, axis = 1)} function applied to each row, output is Series

	\subsubsection{pivot table for dataframe}

\begin{lstlisting}
mean_age_by_class = df.pivot_table(index="class", values="age", 
				aggfunc=numpy.mean) 
\end{lstlisting}

			==> Series of age mean by class

\begin{lstlisting}
statsdf.pivot_table(index="pclass", values=["age", "size"], 
			aggfunc=np.mean)
\end{lstlisting}

			 ==> dataframe with mean age and size by class

	\subsection{Series}

		NB: acts as both a dictionnary and a list

		NB: extends numpy ndarray operation, matlab-like stuffand boolean stuff


		\subsubsection{creation and indexing}
			
			\code{myserie = Series(index = liste1, data = liste2)}

			\code{myserie.index} gets the index as an Indexes object

			\code{.sort\_{}index()} sort by index

			\code{.sort\_'}values()} sort by values

		
		\subsubsection{Operations}
			
			\code{mySerie2 = mySerie / 1000 OR 12 + mySerie OR etc} scalar

			\code{mySerie3 = mySerie * mySerie2} line by line

			\code{.min(), .max(), ...}

			\code{.idxmin()} gets the argmin, ie the index for the lowest value

			\code{.mean()} removes the missing values 

			\code{.isnull} column of boolean as output

			\code{.tolist()}

			\code{.to\_{}dict()} converts to dictionary



\section{Vizualization}

	\subsection{Matplotlib}

		\subsubsection{pyplot basic plots}
	
			\begin{lstlisting}
				import matplotlib.pyplot as plt
			\end{lstlisting}

			\code{plt.style.use("fivethirtyeight")} change style, before creating plot

			\code{plt.scatter(xdata,ydata, color = 'red')}  scatter plot

			\code{plt.show()} has to be added to show the graph

			NB: add several scatter with same x values before show to plot several data as y

			\code{plt.plot(xdata,ydata)} creates a line plot

			\code{plt.bar(xdata,ydata)} creates a bar chart

			\code{plt.axvline(myfloat, color='r')} add verical line 

			\code{plt.barh(ydata,xdata)} creates a horizontal bar chart

			\code{plt.title("montitre")} add a title

			\code{plt.xlabel("monlabel")} add a label for x-axis

			\code{plt.ylabel("monlabel")}

		\subsubsection{pyplot Figure object}

			\code{fig = plt.figure(figsize=(w,h))} sizes in inch

			\code{.figsize(w,h)} size in inches

			\code{.dpi()} dots per inch

			\code{.add\_{}subplot(nb of rows,nb of columns, number of the plot)}

		\subsubsection{pyplot Subplot and Axes object}

			\code{ax = fig.addsubplot(1,1,1)} return an Axes object

			\code{ax.set\_{}xlim([min,max])}} min and max of x axe

			\code{ax.set\_{}ylim([min,max])} min and max of y axe

			\code{ax.scatter(xdata, ydata, color=mycolor, marker=mymarker)} add points

			\code{ax.set\_{}xlabel("xlegend")}

			\code{ax.set\_{}ylabel("ylegend")}

			\code{ax.set\_{}title("titre")}

			==> wrappup: \code{ax.set(xlim =..., ylim=...)}


		\subsection{Plots with pandas}

		\subsubsection{plots on Dataframe object}

			\code{pd.scatter\_{}matrix(df)} plots each series vs the others, diag is histogram

			\code{df.plot(kind='...')} allows to plot each series in the df in one plot

		\subsubsection{histograms in panda (derived from matplotlib)}

\begin{lstlisting}
df.hist(column = ['nomcolonne1','nomcolonne2'],layout=(2,1), grid=False, bins = 20)
\end{lstlisting}

			=>layout gives the position of the 2 graphs
			=>grid indicates if background has a grid
			=>bins number of bars

		\subsubsection{boxplots in panda (derived form matplotlib)}

			boxplot shows quartiles, if col 1 takes categorical values and col2 numerical, you'll get the quartiles of the values taken by col2 for each category in col1 

\begin{lstlisting}
df[['col1', 'col2']].boxplot(by='col1')
\end{lstlisting}


	\subsection{Basemap: doing maps}
	
		map plotting derived of matplotlib, needs to import matplotlib.pyplot

		\code{from mpl\_{}toolkits.basemap import Basemap}

		\code{m = Basemap(projection = 'merc', llcrnrlat = -80, urcrnrlat=80, llcrnrlon=-180,urcrnrlon=180)} instancing with mercator projection, the figures give the min and max

		\code{x, y = m(longitudes, latitudes)} convert from long, to cartesian

		\code{m.scatter(x,y, s = 1)} s is size of the dots

		\code{m.drawcoastlines()} add the coastlines

	\subsection{Seaborn (complement for matplotlib, easy vizualization)}

		simply importing it changes the look of the plots for the better

		==> matplotlib + pandas for exploration

		==> seaborn customized and taylor-made vizualization with simple API

		\code{import seaborn as sns}

		\code{sns.set\_{}style(style = 'dark')} allows to change the style

		\code{sns.axlabel('nameOfXAxis', 'nameOfYAxis')} legend

		\subsubsection{histograms}

			\code{sns.distplot(myseries,kde=False)}

			\code{sns.plt.show()}

			==>kde plot or dont plot the kernel distribution function

		\subsubsection{boxplot}
			\code{sns.boxplot(x=myserie1, y=myserie2, ax =myAxes)}

		\subsubsection{pairplot}
			plots each column against each column in a matrix, histo for the diagonal

			\code{sns.pairplot(df, vars=['nomcolonne1', 'nomcolonne2', 'nomcolonne3'])}


\section{Get web pages and data from API}

	\subsection{Requests: work with API}

		\subsubsection{params and headers}
			are defined in the API documentation
			usually headers are used to authenticate the user (see below)

		\subsubsection{get request}

			\code{req = request.get("http...")} obtain data, there is an optional 

			dictionnary of parameters, and headers

			\code{req.content} gets you the content of the request

			\code{req.headers} dictionary of meta data on the request

			\code{req.headers['content-type']} gives you the content type

			\code{data = req.json()} gets the data in the form of a python object (dict, list of dict etc)
			\code{req.status\_{}code}

			==>200 = ok

			==>401 = not authticated

			==>404 = not found

		\subsubsection{post request}

			can post some jason using the post method

			\code{mydict =\{'name':'some name'\}}
			\code{req = requests.post('http...', json = mydict, headers = ...)} ==> when succesful, the status will be 201

		\subsubsection{modification request}

			\code{request.patch} change parts of an object

			\code{request.put} send the full object to update it

			\code{request.delete} to dlete stuff

		\subsubsection{authentication}
			
			NB: using username + psswd is shut if someones get your script

			token: similar to a password but with defined rights to get accesse through the api. Passed through a header in request, headers is a dictionnary
			
			\code{headers = \{"Authorization": "token thetoken"\}}

			\code{response = requests.get("https://...", headers=headers)}	


\section{Beautifulsoup : parsing html}

	\subsection{create a parser}
		
		\code{resp = requests.get("some simple html page")}

		\code{parser = BeautifulSoup(resp.content, 'html.parser')}

	\subsection{Access stuff}

		\code{parser.body.p.text} accesses the stuff

		\code{parser.find\_{}all("body")[0].find\_{}all("p")} gets a list

		\code{parser.find\_{}all("p", id="first")} filters by id

		\code{parser.find\_{}all("p", class\_{} ="first")} filters by class

		\code{parser.select("#truc")} use CSS selectors (ici id = "truc")

		NB: can do nesting like in CSS too



\section{Accessing databases in Python}

	\subsection{sqlite3: sqlite DB}

	\subsubsection{connecting}

		\code{import sqlite3 as sql}

		\code{conn = sql.connect('mydb.db')} ===> conn is an sqlite3.connection objecct

	\subsubsection{running requests}

		\subsubsection{cursor object}
			\code{cursor = conn.cursor()}

			\code{cursor.execute("query typed as a string")}

			\code{res = cursor.fetchall()} ==>res is a list of tuple

			\code{res = cursor.fetchone()} returns one result + increments the cursor

			\code{res = cursor.fetchmany(n)} returns n results + increments the cursors

	\subsubsection{skip the cursor}
			\code{res = conn.execute("query typed as a string").fetchall())}

	\subsubsection{closing the connection}

		allows to save the changes (sqlite3 only) and free accesses for other users

		\code{conn.close()}

		\subsection{PostgreSQL: psycopg2}
			similar to sqlite3 in the use

		\subsubsection{connecting}
			
			\code{import psycopg2}

			\code{conn = psycopg2.connect("dbname=.... user=...")} ==> need user to handle permissions

		\subsubsection{transaction}
	
			more sophisticated than SQL: when doing changes on a data base, a transaction is open. If some of the changes fail, the all transaction fail. 
			\code{conn.commit()} to save the changes, have to commit

			\code{conn.rollback()} to cancel the changes have tto rollback

			if neither is done, changes remain pending

		\subsubsection{autocommit}
			\code{conn.autocommit = True} changes are saved immediately; necessary to create a database


\section{Maths and Stats in Python}

	\subsection{scypy.stats}

		\code{import scypy.stats as st}

		\code{st.skew(mylist)}

		\code{st.kurtosis(mylist)}

		\code{slope, intercept, r\_{}value, p\_{}value, stderr\_{}slope = st.linregress(x, y)}

		\subsubsection{chi-square}

			\code{chisquare\_{}value, p\_{}value = st.chisquare(observed, expected)} chi-square stuff

			\code{chisq\_{}value, pvalue, df, expected = chi2\_{}contingency(observed)} multi-category chi-square, can take a pandas crosstab \ref{pd.crosstab} as input directly

		\sbsubsection{Working with classic probability laws}

			\code(from scipy.stats import binom) binomial law

			\code(binom.pmf(....)) probability mass function

			\code(binom.cdf(....)) cumulative density function



	\subsection{numpy}

		NB: most numpy function are compatible with pandas objects
		
		\code{np.median(mylist)}

		\code{np.std(mylist)}

		\code{rho, p\_{}value = pearsonr(data1, data2)} correlation and p-value

		\code{cov(data1, data2)[0,1]} covariance, returns a matrix so need indexing


	\subsection{stats in pandas}
		
		\subsubsection{Series and dataframe methodes}

			\code{.mean()}

			\code{.kurtosis()}

			\code{.skew()}

			\code{.std()}

			\code{.median()}

		\subsubsection{crosstab}
			\label{pd.crosstab}

			\code{table = pd.crosstab(df["col1"], [df['col2'],df['col3'...]])} gets a cross table for categorical column: how many of each cat of col2 falls into each cat of col1


	\subsection{random package}

		\code{import random as rdm}

		\code{rdm.seed(p)} sets the seed as p, kills the randomness, good if always wanna use the same pseudo random numbers.

		\code{a = rdm.sample(MyList, n)} gets a sample of size n from a list

		\code{a = rdm.randint(p, q)} random number between p and q

		\code{a = rdm.rand(p,q,r...)} $p*q*r*$ table of random number in $\[0, 1)$, default size (when empty) is 1




